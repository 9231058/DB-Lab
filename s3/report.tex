%%%%%%%%%%%%%%%%%%%%%%%%%%%%%%%%%%%%%%%%%
% University/School Laboratory Report
% LaTeX Template
% Version 3.1 (25/3/14)
%
% This template has been downloaded from:
% http://www.LaTeXTemplates.com
%
% Original author:
% Linux and Unix Users Group at Virginia Tech Wiki
% (https://vtluug.org/wiki/Example_LaTeX_chem_lab_report)
%
% License:
% CC BY-NC-SA 3.0 (http://creativecommons.org/licenses/by-nc-sa/3.0/)
%
%%%%%%%%%%%%%%%%%%%%%%%%%%%%%%%%%%%%%%%%%

%----------------------------------------------------------------------------------------
%	PACKAGES AND DOCUMENT CONFIGURATIONS
%----------------------------------------------------------------------------------------

\documentclass{article}

\usepackage{graphicx} % Required for the inclusion of images

\setlength\parindent{0pt} % Removes all indentation from paragraphs

\renewcommand{\labelenumi}{\alph{enumi}.} % Make numbering in the enumerate environment by letter rather than number (e.g. section 6)

%\usepackage{times} % Uncomment to use the Times New Roman font

%----------------------------------------------------------------------------------------
%	DOCUMENT INFORMATION
%----------------------------------------------------------------------------------------

\title{In the name of God \\ Database Lab \\ Spring 2016} % Title

\author{Parham \textsc{Alvani}} % Author name

\date{\today} % Date for the report

\begin{document}

\maketitle % Insert the title, author and date

\begin{center}
    \begin{tabular}{l r}
        Date Performed: & March 2, 2016 \\ % Date the experiment was performed
        Partners: & Iman Tabrizian \\ % Partner names
    \end{tabular}
\end{center}

% If you wish to include an abstract, uncomment the lines below
% \begin{abstract}
% Abstract text
% \end{abstract}

%----------------------------------------------------------------------------------------
%	SECTION 1
%----------------------------------------------------------------------------------------

\section{Results and Conclusions}
\begin{enumerate}
    \item[1.]
        First query to create table schema:

        CREATE TABLE "persons-1"\\
        (\\
        P\_Id  int identity(1,1),\\
        LastName varchar(255),\\
        FirstName varchar(255),\\
        Address varchar(255),\\
        City varchar(255),\\
        primary key (FirstName, LastName)\\
        );\\

        \begin{enumerate}
            \item %im-2
                SELECT * FROM "persons-1" ORDER BY LastName ASC;
            \item %im-3

                INSERT INTO "persons-1" (LastName,FirstName ,Address ,City ) VALUES ('Hansen','Ola','Timoteivn 10', 'Sandnes');\\
                INSERT INTO "persons-1" (LastName,FirstName ,Address ,City ) VALUES ('Svendson','Tove','Borgvn 23', 'Sandnes');\\
                INSERT INTO "persons-1" (LastName,FirstName ,Address ,City ) VALUES ('Pettersen','Kari','Storgt 20', 'Stavanger');\\
                INSERT INTO "persons-1" (LastName,FirstName ,Address ,City ) VALUES ('Nilsen','Tom','Vingvn 23', 'Stavanger');\\
                SELECT * FROM "persons-1";
            \item %im-4
                begin transaction t1\\

                UPDATE "persons-1" SET phone\_number='0019392' where P\_id=1;\\
                UPDATE "persons-1" SET phone\_number='0019392' where P\_id=2;\\
                UPDATE "persons-1" SET phone\_number='0019392' where P\_id=3;\\
                UPDATE "persons-1" SET phone\_number='0019392' where P\_id=4;\\
                commit transaction t1;\\

                SELECT * FROM "persons-1";\\
            \item %im-5

                SELECT  FirstName,LastName,Full\_address = (case P\_id \\
                when 1 then 'Jomhori' \\
                when 2 then 'Enghelab'\\
                when 3 then 'Felestin'\\
                else 'Ferdowsi'\\
                end ) FROM "persons-1";\\

            \item %im-6

                SET identity\_insert "persons-1" on;\\
                begin transaction t1\\
                INSERT INTO "persons-1" (P\_id, FirstName, LastName, City,Address,phone\_number) VALUES \\
                (7,'Tjessem','Jakob', 'Nissestien 67', 'Sandnes','0017673276');\\
                SELECT * FROM "persons-1" ORDER BY FirstName ASC;\\
                commit transaction t1;\\
            \item %im-7

                declare @temp int\\
                SELECT @temp = max(P\_id) FROM "persons-1";\\
                while @temp > 0\\
                begin\\
                print 'Okay';\\
                set @temp=@temp-1;\\
                end\\

            \item %im-8
                declare @temp int\\
                SELECT @temp = max(P\_id) FROM "persons-1";\\
                while @temp > 0\\
                begin\\
                print 'Okay';\\
                set @temp=@temp-1;\\
                end\\
            \item %im-9

                SET identity\_insert "persons-1" on;\\
                DELETE FROM "persons-1" where FirstName='taylor';\\
                declare @temp nvarchar(255);\\
                SELECT @temp = phone\_number FROM "persons-1" where Firstname='Tjessem';\\
                declare @casted int;\\
                SET @casted = cast(@temp as int);\\
                if @temp < 0011234567\\
                INSERT INTO "persons-1" (P\_id,FirstName,Lastname, Address,City,phone\_number) VALUES (6,'taylor','Jackson','Nisseisten87','Sandnes','0011234567');\\
                else\\
                INSERT INTO "persons-1" (P\_id,FirstName,Lastname, Address,City,phone\_number) VALUES (8,'taylor','Jackson','Nisseisten87','Sandnes','0011234567');\\
                SELECT * FROM "persons-1";\\
        \end{enumerate}
    \item[2.]
        \begin{enumerate}
            \item Creating table schema

                %im-2-1
                CREATE TABLE "students-1"(\\
                name varchar(255),\\
                student\_id int primary key,\\
                grade int\\
                );\\
                %im 2.2
            \item Adding data to table:

                DELETE FROM "students-1";\\
                INSERT INTO "students-1" (name,student\_id,grade) VALUES ('R1',8831047,12);\\
                INSERT INTO "students-1" (name,student\_id,grade) VALUES ('R2',8831043,10);\\
                INSERT INTO "students-1" (name,student\_id,grade) VALUES ('R3',8831031,15);\\
                INSERT INTO "students-1" (name,student\_id,grade) VALUES ('R4',8831051,16);\\
                INSERT INTO "students-1" (name,student\_id,grade) VALUES ('R1',8831012,11);\\
                SELECT * FROM "students-1";\\
                %2.3
            \item Achieving result:

                declare @temp table(\\
                nameb varchar(255),\\
                student\_idb int,\\
                new\_grade int,\\
                old\_grade int\\
                );\\
                UPDATE "students-1" SET "students-1".grade = "students-1".grade + 2 output inserted.name,\\
                inserted.student\_id,inserted.grade,deleted.grade into @temp where "students-1".grade < 15;\\
                SELECT * FROM @temp;\\
        \end{enumerate}
\end{enumerate}
\end{document}
