%%%%%%%%%%%%%%%%%%%%%%%%%%%%%%%%%%%%%%%%%
% University/School Laboratory Report
% LaTeX Template
% Version 3.1 (25/3/14)
%
% This template has been downloaded from:
% http://www.LaTeXTemplates.com
%
% Original author:
% Linux and Unix Users Group at Virginia Tech Wiki
% (https://vtluug.org/wiki/Example_LaTeX_chem_lab_report)
%
% License:
% CC BY-NC-SA 3.0 (http://creativecommons.org/licenses/by-nc-sa/3.0/)
%
%%%%%%%%%%%%%%%%%%%%%%%%%%%%%%%%%%%%%%%%%

%----------------------------------------------------------------------------------------
%	PACKAGES AND DOCUMENT CONFIGURATIONS
%----------------------------------------------------------------------------------------

\documentclass{article}

\usepackage{graphicx} % Required for the inclusion of images

\setlength\parindent{0pt} % Removes all indentation from paragraphs

\renewcommand{\labelenumi}{\alph{enumi}.} % Make numbering in the enumerate environment by letter rather than number (e.g. section 6)

%\usepackage{times} % Uncomment to use the Times New Roman font

%----------------------------------------------------------------------------------------
%	DOCUMENT INFORMATION
%----------------------------------------------------------------------------------------

\title{In the name of God \\ Database Lab \\ Spring 2016} % Title

\author{Parham \textsc{Alvani}} % Author name

\date{\today} % Date for the report

\begin{document}

\maketitle % Insert the title, author and date

\begin{center}
    \begin{tabular}{l r}
        Date Performed: & March 2, 2016 \\ % Date the experiment was performed
        Partners: & Iman Tabrizian \\ % Partner names
    \end{tabular}
\end{center}

% If you wish to include an abstract, uncomment the lines below
% \begin{abstract}
% Abstract text
% \end{abstract}

%----------------------------------------------------------------------------------------
%	SECTION 1
%----------------------------------------------------------------------------------------

\section{Results and Conclusions}
\begin{enumerate}
    \item
        First query to create table schema:

        CREATE TABLE "persons-1"\\
        (\\
        P\_Id  int identity(1,1),\\
        LastName varchar(255),\\
        FirstName varchar(255),\\
        Address varchar(255),\\
        City varchar(255),\\
        primary key (FirstName, LastName)\\
        );\\

        %2.1
    \item
        CREATE TABLE "students-1"(\\
        name varchar(255),\\
        student\_id int primary key,\\
        grade int\\
        );\\
        %2.2
    \item
        DELETE FROM "students-1";\\
        INSERT INTO "students-1" (name,student\_id,grade) VALUES ('R1',8831047,12);\\
        INSERT INTO "students-1" (name,student\_id,grade) VALUES ('R2',8831043,10);\\
        INSERT INTO "students-1" (name,student\_id,grade) VALUES ('R3',8831031,15);\\
        INSERT INTO "students-1" (name,student\_id,grade) VALUES ('R4',8831051,16);\\
        INSERT INTO "students-1" (name,student\_id,grade) VALUES ('R1',8831012,11);\\
        SELECT * FROM "students-1";\\
        %2.3
    \item
        declare @temp table(\\
        nameb varchar(255),\\
        student\_idb int,\\
        new\_grade int,\\
        old\_grade int\\
        );\\
        UPDATE "students-1" SET "students-1".grade = "students-1".grade + 2 output inserted.name,\\
        inserted.student\_id,inserted.grade,deleted.grade into @temp where "students-1".grade < 15;\\
        SELECT * FROM @temp;\\
    \item %2.4
\end{enumerate}
\end{document}
